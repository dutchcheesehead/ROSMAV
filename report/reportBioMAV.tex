\documentclass[a4paper,10pt]{article}

% Wider pages.
\usepackage[a4paper]{geometry}

% Language.
\usepackage[english]{babel}

% Allows the use of \subtitle{...}.
\usepackage{titling}
\newcommand{\subtitle}[1]{%
  \posttitle{%
    \par\end{center}
    \begin{center}\large#1\end{center}
    \vskip0.5em}%
}

% Allows the use of \includegraphics{...}.
\usepackage{graphicx,subfigure}

% Allows the use of \url{...}.
\usepackage{url}

% Seperate paragraphs by an empty line and removes indentation.
\usepackage[parfill]{parskip}

\begin{document}

%%%%%%%%%%%%%%%%%%%%%%%%%%%%%%%%%%%%%%%%%%%%%%%%%%%%%%%%%%%%%%%%%%%%%%%%%%%%%%%
% TTITLEPAGE
%%%%%%%%%%%%%%%%%%%%%%%%%%%%%%%%%%%%%%%%%%%%%%%%%%%%%%%%%%%%%%%%%%%%%%%%%%%%%%%
\title{BioMAV: the present-inspecting robot}

\author{Robin Wellner and Roland Meertens}

\date{Februari 1, 2012}

\maketitle

%%% Robin's part

%%%%%%%%%%%%%%%%%%%%%%%%%%%%%%%%%%%%%%%%%%%%%%%%%%%%%%%%%%%%%%%%%%%%%%%%%%%%%%%
% INTRODUCTION
%%%%%%%%%%%%%%%%%%%%%%%%%%%%%%%%%%%%%%%%%%%%%%%%%%%%%%%%%%%%%%%%%%%%%%%%%%%%%%%
\section{Introduction}

%%%%%%%%%%%%%%%%%%%%%%%%%%%%%%%%%%%%%%%%%%%%%%%%%%%%%%%%%%%%%%%%%%%%%%%%%%%%%%%
% BACKGROUND
%%%%%%%%%%%%%%%%%%%%%%%%%%%%%%%%%%%%%%%%%%%%%%%%%%%%%%%%%%%%%%%%%%%%%%%%%%%%%%%
\section{Background}


%%%%%%%%%%%%%%%%%%%%%%%%%%%%%%%%%%%%%%%%%%%%%%%%%%%%%%%%%%%%%%%%%%%%%%%%%%%%%%%
% MOTIVATION
%%%%%%%%%%%%%%%%%%%%%%%%%%%%%%%%%%%%%%%%%%%%%%%%%%%%%%%%%%%%%%%%%%%%%%%%%%%%%%%
\section{Motivation}
\textbf{World order}
%%%%%%%%%%%%%%%%%%%%%%%%%%%%%%%%%%%%%%%%%%%%%%%%%%%%%%%%%%%%%%%%%%%%%%%%%%%%%%%
% GOAL
%%%%%%%%%%%%%%%%%%%%%%%%%%%%%%%%%%%%%%%%%%%%%%%%%%%%%%%%%%%%%%%%%%%%%%%%%%%%%%%
\section{Goal}
\subsection{Initial goals}

\subsection{Final goal}

%%% Roland's part


%%%%%%%%%%%%%%%%%%%%%%%%%%%%%%%%%%%%%%%%%%%%%%%%%%%%%%%%%%%%%%%%%%%%%%%%%%%%%%%
% PSEUDOCODE
%%%%%%%%%%%%%%%%%%%%%%%%%%%%%%%%%%%%%%%%%%%%%%%%%%%%%%%%%%%%%%%%%%%%%%%%%%%%%%%
\section{Pseudocode}
In this section several of the implemented algorithms are discussed. Every subsection starts with the code in pseudocode and ends with an explanation of the algorithm. 
\subsection{Detection of objects}
\begin{verbatim}
targets = getAllBlobsThatMatchOurTarget
coolDownHeatMap
if dot in heatmap is high enough
	if sum of heatmap is high enough
		inspect present
	else
		go toward average of activation

\end{verbatim}
Our program first determines that targets we have by aquiring all blobs that match our target. 
After this is done our heatmap is cooled down using a xxxxxxx function. 
When a dot in our heatmap is high enough we assume this is our package. 
When the total sum of the heatmap is high enough (there is enough activation) we inspect the present (see section xxxxxx). 
Otherwist we fly towards the average of the activation (see section xxxxxxx)


TODO: add the part where the heatmap is activated again. 
\subsection{Corridor following} 
\begin{verbatim}
targets = getAllBlobsThatMatchOurTarget
myTarget = getTargetLowestOnCamera
if isVeryMuchToTheLeft(myTarget)
	flyVeryHardToTheleft
else if isVeryMuchToTheRight(myTarget)
	flyVeryHardToTheRight
else if isSlightlyToTheLeft(myTarget)
	flySlowlyToTheLeft
else if isSlightlyToTheRight(myTarget)
	flySlowlyToTheRight
else
	flyForward

\end{verbatim}

insert part about that flying towards the nearest? biggest? object that is pretty white results in the following of lights in the hallway. This of course equals to corridor following

Our program first gets all blobs that are matching our target colour. 
The largest blob is chosen as our primary target. 
When this blob is very much to the left or right the drone will turn with a great speed in this direction. 
When this blob is only slightly to the left or right the drone will turn slowly in this direction. 
Otherwise the drone will fly forwards. 

\subsection{Flying towards objects}
\begin{verbatim}
direction = XvalueOfAverageOfActivation
turnTowards(direction)
flyForward
\end{verbatim}

insert part about left-right and middle 

\subsection{Present inspection}
\begin{verbatim}
if sum(activation) > threshold
	if getNewTarget(currentTarget)
		currentTarget = getNewTarget(currentTarget)
	else
		startLanding
\end{verbatim}

As soon as the total activation in the heatmap for the presents exceeds the set threshold the next target is chosen. 
When there is no new target specified in our goals-array the drone will land. 

%%%%%%%%%%%%%%%%%%%%%%%%%%%%%%%%%%%%%%%%%%%%%%%%%%%%%%%%%%%%%%%%%%%%%%%%%%%%%%%
% DIVISION OF LABOUR
%%%%%%%%%%%%%%%%%%%%%%%%%%%%%%%%%%%%%%%%%%%%%%%%%%%%%%%%%%%%%%%%%%%%%%%%%%%%%%%
\section{Project history and division of labour}
The first BioMAV project started in xxxxxx with an enthousiastic group of people. After the initial project was a large succes and obtained the third price in the imav competition on xxxxxinsert datexxxx a second project was started.
For this second project one had to write a sollicitation letter. 
Initially the group consisted of xxxxxx people (Youetta, Rick, Jurriaan, Roland, Robin and xxxxx).  
These people were devided into two teams, the "image processing" team and the "drone control" group. 
The task of the "image processing" team was defined as getting useful features from the data. 
The "drone control" group, consisting of Roland and Youetta, was supposed to use these features to control the drone.  

Initially both groups were both trying to decode the largely undocumented code made by the previous team.
One of the wishes of the group was a code implementation using ROS. This was another part that both teams had to figure out.  
Around june the first implementation of a control program using ROS was finished by Roland. 
This program consisted of a server extension of the original software written by the first BioMAV team.  
The ROS component connected to a socket that listened to calls coming from the ROS component of the software. 
Using a simple syntax it was possible to fly with the ARDrone using ROS.  

Insert some part about other members stopping.

During a meeting in juli it was agreed that this setup was not usefull enough and that another implementation had to be found that only used ROS. 
Unfortunately this meant that Robin and Roland had to start over again, throwing away all old code (including all code used by BioMAV 1).  
After we threw away all old code we started with searching this "easy usable" library that should have been available on the internet. 
Unfortunately, this did cost us a few days. 
Every library we found was eiter not usable, not downloadable (findable) anymore or not uploaded yet (but sounded very promising). 
It was only after a lot of hours searching that one of the librarys suddenly was pullable again. 

insert part about recognizing colours

insert part about not recognizing markers

insert part about problems with small blobs

insert part about decay functions

%%%%%%%%%%%%%%%%%%%%%%%%%%%%%%%%%%%%%%%%%%%%%%%%%%%%%%%%%%%%%%%%%%%%%%%%%%%%%%%
% REFLECTION
%%%%%%%%%%%%%%%%%%%%%%%%%%%%%%%%%%%%%%%%%%%%%%%%%%%%%%%%%%%%%%%%%%%%%%%%%%%%%%%
\section{Reflection}

Although the bioMAV project has been a rough project this year the final result is something that looks very cool. 
To build a program that lets a small UAV autonomously follow bread-containers, corridors and inspect packages is something to be proud of. 

Of course, one starts to think about how much more awesome the initial plans of the complete team were. 
Unfortunately, since more than half of the team decided to stop early, these plans were never finished and kept for the next team. 

Another thing that was very unfortunate was the fact that the whole old BioMAV project was thrown away in the end. 
Although a lot of time was spend in learning how to interpret the old code and extending the project to work with ROS all these work was thrown away. 
Another thing that was not used after a lot of time went into researching this were the AR markers, unfortunately these proved to be impossible for us to implement. 

Of course we were all very contend with the final result, which is very cool. 


%%%%%%%%%%%%%%%%%%%%%%%%%%%%%%%%%%%%%%%%%%%%%%%%%%%%%%%%%%%%%%%%%%%%%%%%%%%%%%%
% BIBLIOGRAPHY
%%%%%%%%%%%%%%%%%%%%%%%%%%%%%%%%%%%%%%%%%%%%%%%%%%%%%%%%%%%%%%%%%%%%%%%%%%%%%%%
\bibliographystyle{plain}
\bibliography{bibliography}

\end{document}
