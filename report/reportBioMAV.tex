\documentclass[a4paper,10pt]{article}

% Wider pages.
\usepackage[a4paper]{geometry}

% Language.
\usepackage[english]{babel}

% Allows the use of \subtitle{...}.
\usepackage{titling}
\newcommand{\subtitle}[1]{%
  \posttitle{%
    \par\end{center}
    \begin{center}\large#1\end{center}
    \vskip0.5em}%
}

% Allows the use of \includegraphics{...}.
\usepackage{graphicx,subfigure}

% Allows the use of \url{...}.
\usepackage{url}

% Seperate paragraphs by an empty line and removes indentation.
\usepackage[parfill]{parskip}

\begin{document}

%%%%%%%%%%%%%%%%%%%%%%%%%%%%%%%%%%%%%%%%%%%%%%%%%%%%%%%%%%%%%%%%%%%%%%%%%%%%%%%
% TTITLEPAGE
%%%%%%%%%%%%%%%%%%%%%%%%%%%%%%%%%%%%%%%%%%%%%%%%%%%%%%%%%%%%%%%%%%%%%%%%%%%%%%%
\title{BioMAV: the present-inspecting robot}

\author{Robin Wellner and Roland Meertens}

\date{Februari 1, 2012}

\maketitle

%%% Robin's part

%%%%%%%%%%%%%%%%%%%%%%%%%%%%%%%%%%%%%%%%%%%%%%%%%%%%%%%%%%%%%%%%%%%%%%%%%%%%%%%
% INTRODUCTION
%%%%%%%%%%%%%%%%%%%%%%%%%%%%%%%%%%%%%%%%%%%%%%%%%%%%%%%%%%%%%%%%%%%%%%%%%%%%%%%
\section{Introduction}

%%%%%%%%%%%%%%%%%%%%%%%%%%%%%%%%%%%%%%%%%%%%%%%%%%%%%%%%%%%%%%%%%%%%%%%%%%%%%%%
% BACKGROUND
%%%%%%%%%%%%%%%%%%%%%%%%%%%%%%%%%%%%%%%%%%%%%%%%%%%%%%%%%%%%%%%%%%%%%%%%%%%%%%%
\section{Background}


%%%%%%%%%%%%%%%%%%%%%%%%%%%%%%%%%%%%%%%%%%%%%%%%%%%%%%%%%%%%%%%%%%%%%%%%%%%%%%%
% MOTIVATION
%%%%%%%%%%%%%%%%%%%%%%%%%%%%%%%%%%%%%%%%%%%%%%%%%%%%%%%%%%%%%%%%%%%%%%%%%%%%%%%
\section{Motivation}
\textbf{World order}
%%%%%%%%%%%%%%%%%%%%%%%%%%%%%%%%%%%%%%%%%%%%%%%%%%%%%%%%%%%%%%%%%%%%%%%%%%%%%%%
% GOAL
%%%%%%%%%%%%%%%%%%%%%%%%%%%%%%%%%%%%%%%%%%%%%%%%%%%%%%%%%%%%%%%%%%%%%%%%%%%%%%%
\section{Goal}
\subsection{Initial goals}

\subsection{Final goal}

%%% Roland's part


%%%%%%%%%%%%%%%%%%%%%%%%%%%%%%%%%%%%%%%%%%%%%%%%%%%%%%%%%%%%%%%%%%%%%%%%%%%%%%%
% PSEUDOCODE
%%%%%%%%%%%%%%%%%%%%%%%%%%%%%%%%%%%%%%%%%%%%%%%%%%%%%%%%%%%%%%%%%%%%%%%%%%%%%%%
\section{Pseudocode}
In this section several of the implemented algorithms are discussed. Every subsection starts with the code in pseudocode and ends with an explanation of the algorithm. 
\subsection{Detection of objects}
\begin{verbatim}
targets = getAllBlobsThatMatchOurTarget
coolDownHeatMap

\end{verbatim}
\subsection{Corridor following} 
\begin{verbatim}

\end{verbatim}
\subsection{Flying towards objects}
\begin{verbatim}

\end{verbatim}
\subsection{Present inspection}

%%%%%%%%%%%%%%%%%%%%%%%%%%%%%%%%%%%%%%%%%%%%%%%%%%%%%%%%%%%%%%%%%%%%%%%%%%%%%%%
% DIVISION OF LABOUR
%%%%%%%%%%%%%%%%%%%%%%%%%%%%%%%%%%%%%%%%%%%%%%%%%%%%%%%%%%%%%%%%%%%%%%%%%%%%%%%
\section{Project history and division of labour}
The first BioMAV project started in xxxxxx with an enthousiastic group of people. After the initial project was a large succes and obtained the third price in the imav competition on xxxxxinsert datexxxx a second project was started.
For this second project one had to write a sollicitation letter. 
Initially the group consisted of xxxxxx people (Youetta, Rick, Jurriaan, Roland, Robin and xxxxx).  
These people were devided into two teams, the "image processing" team and the "drone control" group. 
The task of the "image processing" team was defined as getting useful features from the data. 
The "drone control" group, consisting of Roland and Youetta, was supposed to use these features to control the drone.  

Initially both groups were both trying to decode the largely undocumented code made by the previous team.
One of the wishes of the group was a code implementation using ROS. This was another part that both teams had to figure out.  
Around june the first implementation of a control program using ROS was finished by Roland. 
This program consisted of a server extension of the original software written by the first BioMAV team.  
The ROS component connected to a socket that listened to calls coming from the ROS component of the software. 
Using a simple syntax it was possible to fly with the ARDrone using ROS.  

Insert some part about other members stopping.

During a meeting in juli it was agreed that this setup was not usefull enough and that another implementation had to be found that only used ROS. 
Unfortunately this meant that Robin and Roland had to start over again, throwing away all old code (including all code used by BioMAV 1).  
After we threw away all old code we started with searching this "easy usable" library that should have been available on the internet. 
Unfortunately, this did cost us a few days. 
Every library we found was eiter not usable, not downloadable (findable) anymore or not uploaded yet (but sounded very promising). 
It was only after a lot of hours searching that one of the librarys suddenly was pullable again. 


%%%%%%%%%%%%%%%%%%%%%%%%%%%%%%%%%%%%%%%%%%%%%%%%%%%%%%%%%%%%%%%%%%%%%%%%%%%%%%%
% REFLECTION
%%%%%%%%%%%%%%%%%%%%%%%%%%%%%%%%%%%%%%%%%%%%%%%%%%%%%%%%%%%%%%%%%%%%%%%%%%%%%%%
\section{Reflection}











%%%%%%%%%%%%%%%%%%%%%%%%%%%%%%%%%%%%%%%%%%%%%%%%%%%%%%%%%%%%%%%%%%%%%%%%%%%%%%%
% BIBLIOGRAPHY
%%%%%%%%%%%%%%%%%%%%%%%%%%%%%%%%%%%%%%%%%%%%%%%%%%%%%%%%%%%%%%%%%%%%%%%%%%%%%%%
\bibliographystyle{plain}
\bibliography{bibliography}

\end{document}
